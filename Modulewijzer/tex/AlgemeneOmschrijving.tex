\section{Algemene omschrijving}
	The overall goal of the course is to provide a detailed answer to the questions:	
	\begin{itemize}
	\item what are the kinematics equations?
	\item how does a physical simulation work?
	\end{itemize}
	
	In this section we discuss further the full breadth of what the course covers, plus the desired level of skills achieved by the students. 
	
	\subsection{Introduction}
		Simulation of the physical world is one of the most powerful fields of application of modern programming disciplines. Thanks to such simulations it is possible to set up virtual experiments, build virtual reality worlds, and many more modern complex applications. \\
		
		Since analytical solutions are very rarely available, building such simulations requires the use of approximation techniques, which themselves are also applied to many other fields such as computer vision, artificial intelligence, and robotics. \\

		The goal of the course is to provide a definition of the physical laws of kinematics, and of numerical approximations of complex dynamics. Moreover, we shall learn how to translate this knowledge into a working physical simulator. \\
				
	\subsection{Relationship with other teaching units}
		This module builds over all modules of programming, and is also strongly connected with previous knowledge about mathematics, and linear algebra. \\
	
	\subsection{Learning tools}
		Obligatory:
		\begin{itemize}
			\item Presentations and sources presented during lectures (found on N@tschool);
			\item Assignments to work on during practicums (found on N@tschool);
			\item Text editors: Emacs, Notepad++, Visual Studio, Xamarin Studio, etc.
		\end{itemize}
		Facultative:
		\begin{itemize}
			\item Book: Game Physics, author: David Eberly
			\item Book: Friendly F\# (Fun with game physics), authors: Giuseppe Maggiore, Marijn Tamis, Giulia Costantini
		\end{itemize}
