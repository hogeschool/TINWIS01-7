\section{Testing and evaluation}
	In this section we discuss the testing procedure of this course, and the grading criteria.
	
	\subsection{Overall description}
		
		This module is tested with a series of practical assignments that are checked in an oral assessment where you have to explain how and why they work. These assignments can be found on N@tschool. \\

		Foreword and notes:
		\begin{itemize}
			\item The practical assignments determine the final grade.
			\item The practical assignments are made up of elements of an interpreter or a compiler which is either incomplete or wrongly built. The students task is that of extending or fixing such elements.
			\item The practical assignments can be made in pairs. The oral assessment is individual.
			\item The practical assignments must contain extensive, individually written documentation.
		\end{itemize}
		\ \\
		
		This manner of examination is chosen for the following reasons:
		\begin{itemize}
			\item By reading existing sources students must read and reason about code (learning goals \textit{analysis} and \textit{advice}).
			\item By correcting or extending the sources students must write code (learning goals \textit{design} and \textit{realisation}).
			\item By writing documentation students must communicate about their code (learning goal \textit{communication}).
		\end{itemize}

		\ \\
		The grade of each practicum assignment is determined by:
		\begin{itemize}
			\item The correctness of the underlying physical and approximation rules (60\%).
			\item Completeness and clarity of the documentation (40\%).
		\end{itemize}

	\subsection{Assignments}
	
		\paragraph{Assignment 1 - linear motion}
			Students must define an Euler-integrated simulation over position, velocity (or linear momentum) and force.
			
		\paragraph{Assignment 2 - rotational motion}
			Students must define an Euler-integrated simulation over rotation, angular velocity (or angular momentum) and torque.
		
		\paragraph{Assignment 2B - Gram-Schmidt ortho-normalization}
			Students must define a Gram-Schmidt ortho-normalization.
			
		\paragraph{Assignment 2C - quaternions}
			Students must replace rotation matrices with quaternions.

		\paragraph{Assignment 3 - Runge-Kutta integration}
			Students must replace the Euler integration with a Runge-Kutta integration of order 4.

	\subsection{Grades}
		Assignments 1, 2, and 3 are obligatory. With these assignments the maximum grade possible is a seven. The other assignments are all optional and give each one and a half additional point.
		
\begin{tabularx}{\textwidth}{|>{\columncolor{lichtGrijs}} p{3cm}|X|}
	\hline
	\textbf{Assignments:} & \textbf{Value in grades} \\
	\hline
	1, 2, 3 & 7 \\
	\hline
	2B & 1.5 \\
	\hline
	2C & 1.5 \\
	\hline
\end{tabularx}
		

	\subsection{Deadlines}
		The assignments must be handed in, printed on paper, before the end of the last lecture on week 8 of the period. \textit{Herkansing} can be done by handing in the assignments before the end of week 1 of the following period (that would be right after the summer holiday).

	\subsection{Feedback}
		It is possible to discuss the assignments (and their evaluation) during the practicum lectures, or during week 10 of the period (the same holds for the \textit{herkansing}, but in reference to the following period).
		