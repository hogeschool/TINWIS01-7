\documentclass{beamer}
\usetheme{Warsaw}
\usepackage{nhtvslides}
\usepackage{graphicx}
\usepackage{listings}
\lstset{language=CAML,
basicstyle=\ttfamily\footnotesize,
frame=shadowbox,
breaklines=true}
\usepackage[utf8]{inputenc}

\title{Building a physics engine - part 5: various forces}

\author{Dr. Giuseppe Maggiore}

\institute{NHTV University of Applied Sciences \\ 
Breda, Netherlands}

\date{}

\begin{document}
\maketitle

\begin{frame}{Table of contents}
\tableofcontents
\end{frame}

\section{Gravity}
\begin{slide}{Gravity}{Gravity}{
\item The easiest :)
\item $F_G = \frac{G m_1 M_2}{r^2}$ $G=6.67384 10^{-11} m^3 kg^{-1} s^{-2}$
\item On the surface of a planet, $F=ma$, where $a = \frac{G m_2}{r^2}$; for example, $a = g = 9.81$
}\end{slide}

\section{Friction}
\begin{slide}{Friction}{Friction}{
\item Friction is related to the \textit{normal force}
\item $F_N = m g \cos \theta$
\item $F_F = \mu F_N$, where $\mu$ depends on the materials
\item When not moving, static friction applies $\mu_S$
\item When moving, dynamic friction applies $\mu_K$
\item $\mu_K < \mu_S$
}\end{slide}

% TABLE OF FRICTIONS AND MATERIALS

\section{Springs}
\begin{slide}{Springs}{Springs}{
\item Hooke's Law
\item $F_S = -k \Delta l$
}\end{slide}

% FIG 3.7

\section{Centripetal force}
\begin{slide}{Centripetal force}{Centripetal force}{
\item During a circular movement
\item Force towards the center of the circle
\item $F_C = \frac{m v^2}{r}$
}\end{slide}

\section{Projectiles}
\begin{slide}{Projectiles}{Projectiles}{
\item A projectile is influenced by
\begin{itemize}
\item Gravity
\item Aerodynamic drag
\item Laminar and turbulent flow
\item Wind effects
\item Spin effects
\item Projectile geometry and mass
\end{itemize}
}\end{slide}

\begin{slide}{Projectiles}{Aerodynamic drag}{
\item $F_D = \frac{1}{2} \rho v^2 A C_D$
\begin{itemize}
\item $\rho$ is the fluid density
\item $A$ is the body area
\item $C_D$ is the drag coefficient
\end{itemize}
}\end{slide}

\begin{slide}{Projectiles}{$C_D$}{
\item Drag is a force applied against motion $-\frac{v}{|v|}$
\item The drag coefficient depends on the \textit{Reynolds number}
\item $C_D = f(Re)$ for some complex function $f$
\item $Re = \frac{\rho v L}{\mu}$
\begin{itemize}
\item $L$ is the projectile length
\item $\mu$ is the drag coefficient
\end{itemize}
}\end{slide}

% FIG 5-6
% FIG 5-3

\begin{slide}{Projectiles}{Air density}{
\item Depends on altitude
}\end{slide}

% TAB 5-4

\begin{slide}{Projectiles}{Laminar vs turbulent flow}{
\item Depending on the speed, the air travels over the surfaces of the projectile
\item At low Reynolds numbers, flow is laminar
\item At around $Re = 250000$ the switch occurs (for a golf ball and similar projectiles)
\item At high Reynolds numbers, flow is turbulent
\item We store at least two $C_D$ coefficients per object
}\end{slide}

\begin{slide}{Projectiles}{Wind}{
\item Wind simply changes the \textit{apparent velocity}
\item All other computations simply use the apparent velocity
\item $v_{\text{apparent}} = v - v_{\text{wind}}$
}\end{slide}

\begin{slide}{Projectiles}{Spin effects}{
\item When an object is spinning, the velocity difference between the top and bottom surfaces given by spinning causes acceleration
\item This is known as the \textit{Magnus Effect} or \textit{Robin's Effect}
}\end{slide}

% FIG 5-14

\begin{slide}{Projectiles}{Spin effects}{
\item $F_M = \frac{1}{2} \rho v^2 A C_L$
\item $C_L = \frac{r w}{v}$ for a sphere
\item $C_L = \frac{2 \pi r w}{v}$ for a cylinder
\item Direction of force is $\frac{v \times w}{|v \times w|}$
}\end{slide}

\begin{slide}{Projectiles}{Bullets}{
\item Rotation axis is horizontal
\item Almost zeroes $\frac{v \times w}{|v \times w|}$
\item Magnus Effect can be ignored
\item $C_D = 0.3$ is a reasonable constant
}\end{slide}

% TAB 5-6

\begin{slide}{Projectiles}{Cannonballs}{
\item Start with a velocity between $260m/s$ and $344m/s$
\item Weight is between $2kg$ and $10kg$
\item $C_D = 0.3$ is a reasonable constant
}\end{slide}


\section{Assignment}
\begin{slide}{Assignment}{Assignment}{
\item Before the end of next week
\item Group-work archive/video on Natschool or uploaded somewhere else and linked in your report
\item Individual report by each of you on Natschool
\item Add a personalized selection of forces to your simulator
}\end{slide}

\begin{frame}{That's it}
\center
\fontsize{18pt}{7.2}\selectfont
Thank you!
\end{frame}

\end{document}


\begin{slide}{SECTION}{SLIDE}{
\item i
}\end{slide}

\begin{frame}[fragile]{SLIDE}
\begin{lstlisting}
CODE
\end{lstlisting}
\end{frame}

\begin{frame}{SLIDE}
\center
%\includegraphics[height=5cm]{Pics/recursive_multiplier.png}
\end{frame}
